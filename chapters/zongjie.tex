\chapter{总结与展望}
\section{工作总结}
现代服务业是中国经济发展战略中的重要组成部分,也是衡量一个国家经济发展水平的重要标志,
人工智能技术的更新正不断影响着人们的生活,近年来愈来愈多的用户采用人机对话的方式来高效地获取优质服务。
跨界服务网络是各类跨界服务集成的支撑系统,将跨越不同行业、组织、价值链等边界的服务进行深度融合和模式创新,
为用户提供多维度、高质量、富价值的跨界服务,成为现代服务业发展的重要创新途径,
相比传统的服务集成,跨界服务融合需开展模式、生态、环境、质量、价值等多维深度融合,
导致内部服务种类繁多、数量庞大,用户在进入系统后,面对如此量级的服务,用户在服务检索时需要花费大量时间,
同时也无法快速查询到与自己意图匹配的服务,因此如何提升用户检索服务时的体验成为问题。

借助人机对话的思想,本文在跨界服务平台中引入服务智能调用引擎。用户进入平台以后,可以输入带有意图的语句,如“查询成都开往杭州的火车票”,
服务智能调用引擎接受语句以后进行包含服务分类、接口分类和参数提取(语义槽填充)三项任务的语义理解,识别并找出系统内部与之匹配的服务,
从句子中提取服务调用必需的参数完成调用返回结果,从而解决了用户检索服务困难的问题。
服务智能调用引擎的核心是语义理解,语义理解模型的性能优劣是衡量调用引擎的智能化程度的重要标准,语义理解的结果也直接影响了服务检索和后续的服务调用。
当前,中文语义理解任务有着诸多困境,如特定下游任务缺乏充分的相关数据集,用户表述意图模糊、随意不规范,一词多义等,因此相关研究受到密切关注,
同时语义理解作为自然语言处理领域的基石性任务,它的研究具有重大科研意义和应用用途。

本文的主要研究工作和创新点如下:

1.针对跨界服务领域中文语料匮乏的现状,本文在公开数据集基础上借助搜索引擎和课题组成员人工补充,构建了跨界服务领域的语义理解数据集。
扩充后的数据集包含8个服务类别,每条语料都包含对应的服务类别、接口类别和语义槽标注,共计19145条数据。

2.本文提出了端到端的服务分类、接口分类和参数提取三项任务交互式联合识别模型,并引入BERT作为预训练模型进行微调和和引入知识库增强短文本输入的语义,
与三项任务独立处理的模型、基于word2vec的模型做了对照实验,验证了模型的可行性和优越性,其中表现最好的BERT-co-interactive模型Sentence Acc达到91.47\%。

3.为将算法应用到实际,依托国家重点研发计划专项《现代服务业共性关键技术研发及应用示范》的子课题《跨界服务集成方法与支撑载体》的原型系统JTangYdrail,
设计了服务智能调用引擎的系统架构,让引擎能与现有平台相结合。

\section{未来展望}
本文提出了引入预训练模型的交互式联合识别模型,在跨界服务领域相关数据集上有了不错的表现,并提出了服务智能调用引擎的系统架构,但现阶段
只是原型系统,没有进入实际生产环境。未来有以下几点改进方向:

1.目前系统内数据集总量有限,且包含服务类别仅有8类,深度学习模型的优异很大程度取决于训练数据集是否充分,如果可以通过各种方式扩充数据集,
那本文的模型一定可以获取更好的优化。

2.最优,有不少研究者将图神经网络应用于文本分类领域并获得了不错的效果,后续可以考虑采用类似的模型以期获得更好的性能。

3.co-interactive模型本身较为复杂,引入BERT以后参数总量大大增加,不利于训练和移植,可以考虑采用较轻量级的模型来改进。

