\begin{englishabstract}
The cross-border service network can break through the traditional organization, business and domain boundaries, 
and has become the main tool to promote cross-enterprise, cross-domain and cross-industry cooperation. 
However, the cross-border service network has a wide variety and huge number of internal services,
users have to spend a lot of cost in service retrieval. 
Therefore, how to improve user's experience becomes a problem. With the help of man-machine dialogue, 
this paper introduces an intelligent service invocation engine into the cross-border service network system. 
The engine can identify the matching service within the system according to the sentence passed in by user
and extract the necessary parameters from the sentence to complete the invocation and return the result. 
The core of the engine is semantic understanding model,whose performance is an important criterion to measure the degree of engine intelligence, 
and it also directly affects subsequent service calls. At present, Chinese semantic understanding tasks have many challenges, 
such as lack of sufficient data sets for specific downstream tasks, vague user expression intentions, polysemous words, etc. 
At the same time, semantic understanding is the cornerstone of natural language processing, so its research has great significance and application prospects. 

In view of the lack of Chinese corpus in the field of cross-border services, 
this paper builds a semantic understanding data set in the field of cross-border services 
based on the public data set with the help of search engines and artificial supplements by members of the research team. 
The expanded data set contains 8 service categories, and each corpus contains corresponding service category, 
interface category and semantic slot annotations. At the same time, 
this paper proposes an end-to-end interactive joint recognition model, 
introducing bert as a pre-training model and a knowledge base to enhance the semantics of short text input.
we do control experiments with other models to verify the feasibility and superiority of the model,
Among them, the best-performing bert-co-interactive model's Sentence Acc reached 91.47\%. 
In order to apply the algorithm to practice, 
relying on the prototype system JTangYdrail of the sub-project "Cross-border Service Integration Methods and Support Carriers" of the National Key Research and Development Program "Modern Service Industry Common Key Technology Research and Development and Application Demonstration", 
The system architecture of service intelligent call engine was designed,which allows the engine to be integrated with the existing platform.

\englishkeywords{Natural Language Understanding,Text Classification, Slot Filling, Intent Detection,BERT}
\end{englishabstract}
