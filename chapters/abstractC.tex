\begin{abstract}
跨界服务网络突破传统的组织,业务和域边界,已成为促进跨企业,跨领域和跨行业合作的主要工具。 
但跨界服务网络内部服务种类繁多、数量庞大,用户在进入系统后,面对如此量级的服务,在服务检索时需要花费大量成本,
因此如何提升用户检索服务时的体验成为问题。
借助人机对话的思想,本文在跨界服务网络系统中引入智能服务调用引擎,
引擎根据用户传入的语句能够识别系统内部与之匹配的服务,
从句子中提取服务调用必需的参数完成服务调用并返回结果。
引擎的核心是语义理解模型,其性能优劣是衡量引擎智能化程度的重要标准,也直接影响了后续的服务调用。
当前,中文语义理解任务有着诸多挑战,如特定下游任务缺乏充分的数据集,用户表述意图模糊、随意不规范,一词多义等,
同时语义理解是自然语言处理领域的基石,因此它的研究具有重大意义和应用前途。

针对跨界服务领域中文语料匮乏的现状,本文在公开数据集基础上借助搜索引擎和课题组成员人工补充,构建了跨界服务领域的语义理解数据集。
扩充后的数据集包含8个服务类别,每条语料都包含对应的服务类别、接口类别和语义槽标注,共计19145条数据。
同时,本文提出了端到端的服务分类、接口分类和参数提取三项任务交互式联合识别模型,引入bert作为预训练模型和引入跨界服务相关知识库增强短文本输入的语义,
与三项任务独立处理基于word2vec的模型做了对照实验,验证了模型的可行性和优越性,其中表现最好的bert-co-interactive模型Sentence Acc达到91.47\%。
为将算法应用到实际,依托国家重点研发计划专项的子课题《跨界服务集成方法与支撑载体》的原型系统JTangYdrail,
设计了服务智能调用引擎的系统架构,让引擎能与现有平台相结合。

\keywords{自然语言理解,文本分类,语义槽填充,意图识别,BERT}
\end{abstract}
