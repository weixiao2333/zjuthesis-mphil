\begin{abstract}
人工智能技术的更新正不断影响着人们的生活,近年来愈来愈多的用户采用人机对话的方式来高效地获取优质服务。
跨界服务平台是各类跨界服务集成的支撑系统,将跨越不同行业、组织、价值链等边界的服务进行深度融合和模式创新,
为用户提供多维度、高质量、富价值的跨界服务,成为现代服务业发展的重要创新途径,
相比传统的服务集成,跨界服务融合需开展模式、生态、环境、质量、价值等多维深度融合,
导致内部服务种类繁多、数量庞大,用户在进入系统后,面对如此量级的服务,用户在服务检索时需要花费大量时间,
同时也无法快速查询到与自己意图匹配的服务,因此如何提升用户检索服务时的体验成为问题。

借助人机对话的思想,本文在跨界服务平台中引入服务智能调用引擎。用户进入平台以后,可以输入带有意图的语句,如“查询成都开往杭州的火车票”,
服务智能调用引擎接受语句以后进行包含服务分类、接口分类和参数提取(语义槽填充)三项任务的语义理解,识别并找出系统内部与之匹配的服务,
从句子中提取服务调用必需的参数完成调用返回结果,从而解决了用户检索服务困难的问题。
服务智能调用引擎的核心是语义理解,语义理解模型的性能优劣是衡量调用引擎的智能化程度的重要标准,语义理解的结果也直接影响了服务检索和后续的服务调用。
当前,中文语义理解任务有着诸多困境,如特定下游任务缺乏充分的相关数据集,用户表述意图模糊、随意不规范,一词多义等,因此相关研究受到密切关注,
同时语义理解作为自然语言处理领域的基石性任务,它的研究具有重大科研意义和应用用途。

针对跨界服务领域中文语料匮乏的现状,本文在公开数据集基础上借助搜索引擎和课题组成员人工补充,构建了跨界服务领域的语义理解数据集。
扩充后的数据集包含8个服务类别,每条语料都包含对应的服务类别、接口类别和语义槽标注,共计19145条数据。
同时,本文提出了端到端的服务分类、接口分类和参数提取三项任务交互式联合识别模型,并引入bert作为预训练模型进行微调,
与三项任务独立处理的模型、基于word2vec的模型做了对照实验,验证了模型的可行性和优越性,其中表现最好的bert-co-interactive模型Sentence Acc达到91.47\%。
为将算法应用到实际,依托国家重点研发计划专项《现代服务业共性关键技术研发及应用示范》的子课题《跨界服务集成方法与支撑载体》的原型系统JTangYdrail,
设计了服务智能调用引擎的系统架构,让引擎能与现有平台相结合。

\keywords{跨界服务,文本分类,语义槽填充,意图识别,BERT}
\end{abstract}
