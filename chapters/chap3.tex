\chapter{绪论}

\section{课题背景}

依托以互联网为代表的信息技术的高速发展,我国社会当前正处在由传统服务业向现
代服务业全面升级的重要历史进程中。充分利用和结合现代先进的信息技术,提供信息和
知识更加密集、附加值更高的服务是现代服务业的基本要求。互联网作为现代信息服务的
载体,从早期简单的门户网站、搜索引擎,发展到社交网站、即时通信,再到移动搜索、LBS
等移动互联网应用的风靡,在产业规模持续扩大的同时,也不断向各行各业渗透,从早期
的传媒、游戏等行业,到娱乐、零售行业,再到金融、教育和医疗等行业,影响范围还在不
断继续扩大\cite{王晓玲2015我国现代服务业借力}。由此可见,未来服务的基础形态一定是基于互联网的,各行业通过互联网来
提供他们的服务是大势所趋。

跨界服务将跨越不同行业、组织、价值链等边界的服务进行深度融合和模式创新,为用户提供多维度、高质量、富价值的跨界服务,成为现代服务业发
展的重要创新途径。然而,目前针对跨界服务本质规律认知、跨界服务融   
合理论、工程设计方法与运行载体等方面仍然缺少系统研究,缺少充分理
论指导的跨界服务融合实践呈现一定的盲目性,极大影响我国现代服务业
的创新发展。

早期的基于网络的应用服务通常构建于一组相互联系的Web Service 之上。根据W3C
的定义,Web Service 是指用于支持网络内机器之间互操作的软件系统,它通常包含一个机
器可处理的接口描述(一般是WSDL),其它系统按照其接口描述通过SOAP 消息与它进行
交互\cite{verborgh2018web}。Web Service 包含一系列标准化的规范和技术用以支持基于Web 的应用的集成,包
括XML, SOAP, WSDL 和UDDI 等。

Web Service 在早期许多大型企业级软件应用中使用广泛,但由于其相关标准和技术过
于复杂等原因,在如今的新兴互联网应用中已经很少使用了。目前Web API,作为一种更
加灵活和轻量级的解决方案,摒弃了WS 系列的相关复杂标准,得到了广泛的应用\cite{zaveri2017smartapi}。

Web API 如今是Web、物联网、云计算和机器学习应用等的基石\cite{tan2016service}。在Web 应用领域,
随着前后端分离架构的普及,很多Web 应用的界面背后都是由相关的Web API 在直接支
持动态的数据获取和功能访问。在物联网应用中,智能设备和终端也经常会用到诸如广告、
社交网络、消息和支付等Web API\cite{gorla2014checking,viennot2014measurement}。在云计算领域,云服务提供商提供的诸如计算、存
储、消息和数据库等基础服务,都是以Web API 形式供使用者按需调用。在机器学习应用
中,也有许多诸如谷歌翻译这样的Web API,使得开发者无需收集数据、训练模型,也能
直接使用顶尖的图像分类、语音识别和机器翻译等能力。

正是意识到Web API 具有的无限潜能,许多公司选择成立自己的开放平台,开放自家
的部分Web API,一起共建更大的服务生态,实现开放共享、互利共赢。作为最大的Web
API 收录网站,ProgramWeb 网站已经收录了12453 个Web API 和4593 个组合服务。当然
还有许多收费的、企业内部使用的Web API 远远没有收录。Web API 背后所代表的数据和资源,已经促成了所谓API 经济的形成。出现了以京东万象、聚合数据等为代表的各种API
商店。许多供应商仅仅靠提供Web API 就能获得可观的收入,比如SalesForce, AWS。[7]
尽管Web API 目前已经得到了非常广泛的应用,然而在使用Web API 的过程中仍然存
在很多问题。

\begin{figure}[htbp]
  \centering
  \includegraphics[scale=1]{./images/hospitalRunningModel.jpg}
  \caption{传统医院经营模式下,各家医院标准不一}
  \label{fig:hospitalRunningModel}
\end{figure}

首先,Web API 的描述问题:许多Web API 缺乏有效的描述文档,这给其使用者带来
了很大的障碍。另外,Web API 缺乏统一的描述规范,导致各个公司、组织等都在使用自定
义的、不一致的描述方式,这对于各企业、组织间Web API 的集成和交互式非常不利的,如图\ref{fig:hospitalRunningModel}所展示的,各个地方的医院系统标准不一。再
有就是现在的Web API 描述文档,都是针对人类用户的,缺乏机器可理解性,这对于Web
API 的高效应用,诸如自动组合等方面都是很不利的。

其次,Web API 当前的使用体验问题:用户需要仔细阅读相应的说明文档,然后往往
还需要根据所提供的Web API 的请求和响应等,结合自己的实际需要,进行一层定制或者
包装工作,要正确的使用一个Web API 来获得满足自己需求的结果其实并不容易。

然后,Web API 的选择问题:在很多功能类似的服务中选择一个合适的对很多用户来
说是个挑战。

最后,Web API 的演化问题:服务提供者经常需要升级他们的服务,尽管他们力求做
到向后兼容,但往往还是会或多或少的破坏原来的Web API 接口,服务使用者不得不被迫
修改他们的应用代码以适应新的Web API 的要求,这经常是一项枯燥又繁琐的工作。

本文从服务使用者的角度出发,为解决上述Web API 使用过程中的问题,提出了服务
映射的方法,并开发了相应的系统来支持用户更好的使用服务。

\section{研究意义}
现代服务业是中国经济发展战略中的重要组成部分,也是衡量一个国家经济发展水平
的重要标志。推动现代服务业的发展壮大已成为当前中国经济发展的重要目标和关键动力。
服务计算作为现代服务业发展的重要技术基础,需要结合现代服务业发展的具体场景和具
体问题进行深入的研究和应用。同时,近年来,数字经济已成为全球经济的重要驱动力,“工业+技术”的无限融合不断为市场经济注入新的动力。
在这一发展的浪潮中,企业将大多数业务以Web服务的形式部署在网络中,并为实现复杂业务提供了最基本的功能单元。
最初,这种复杂的业务模型通常是在企业内部实现的,即不同部门根据预定的任务划分提供自己的服务,最终实现复杂而完整的产品。 
但是,随着开发人员要求日益复杂和全球分工明确,单个企业可以提供的服务将不再能满足市场需求。 因此,需要跨企业边界的服务流程来实现业务增值。

本文依托于国家重点研发计划专项《现代服务业共性关键
技术研发及应用示范》的子课题《跨界服务集成方法与支撑载体》,围绕在研究跨界服务集
成和交互过程中发现的Web API 描述文档缺乏、描述方式不一致等导致的集成困难、难以
交互,以及服务使用者使用Web API 过程中遇到的门槛较高、难以上手,需要根据自身需
求进行包装和定制操作,同类服务难以选择,以及Web API 演化导致的应用失败等实际问
题进行了深入的研究和分析,提出了一套以服务映射为基础的解决方案,并
开发了对应的原型系统,不但能够有效解决跨系统的Web API 集成问题,而且提出了以用户为中心的Web API 的使用方式,能够大大简化和改善用户使用Web API 的流程和体验,
还能够避免Web API 演化带来的应用失败的问题,因此本文的研究具有重要的意义。如图\ref{fig:yilianti}
所示,医联体环境下医院经营模式,集成了海量的服务,对用户暴露统一的接口,大大简化了用户的操作。



具体来说,对于一个大型的跨界服务接入平台,当接入的服务数量到达一定阈值时,会出现大量同类型的服务重复接入的情况,比如每一家医院
都有自己的药品查询服务,这一类服务每家医院的名称可能不同,参数可能略有差异,但输入的语义信息都十分接近,可以看作同质服务,
随着接入医院数量的增多,这无疑是一件繁琐且冗余的工作,增加了开发人员的负担而且会有程序漏洞的风险;而且,每个具体的服务参数不同,
对于用户来说,同类型服务下每一个服务就要输入一组参数,而这些参数往往差异不大,简化用户操作无疑势在必行。基于此,
我们提出了服务映射的概念,对于某一类型的服务,在平台内部提出标准服务的概念,标准服务有自己的属性和参数,可在平台内运行,
将该类型同质异构服务自动映射到标准服务,从而快速接入服务,去除繁琐的冗余工作,节约开发成本,
对外暴露统一的标准服务,方便用户,提升用户体验,用户在调用执行标准服务时系统自动映射到实体服务中。

\begin{figure}[htbp]
  \centering
  \includegraphics[scale=1]{./images/yilianti.jpg}
  \caption{医联体环境下医院经营模式图}
  \label{fig:yilianti}
\end{figure}

 同时,在当前情况下,当企业想要实施复杂的业务时,如果没有单个Web服务可以满足用户所需的功能,则应该有可能将现有服务组合在一起以满足请求。
 首先需要分析如何将业务划分为几个模块,
 以及每个模块需要什么原子服务。之后,他需要构建服务流程并基于QoS进行服务选择。
 上述两个步骤中的每一个都十分繁琐,并且需要非常丰富的领域专家知识。显而易见,对于这样的一个个流程,当数量达到
 一定阈值时,也会出现上述和标准服务一样的问题:各个服务流程提供商对解决的同一问题呈现出了不同的流程细节,
随着接入流程提供商数量的增多,这无疑是一件繁琐且冗余的工作,增加了开发人员的负担而且会有程序漏洞的风险;而且,每个具体的流程参数、步骤不同,
对于用户来说,同类型流程下每一个服务就要输入一组参数,而这些参数往往差异不大,简化用户操作无疑势在必行。
 本文将同类型(同质)实体流程抽象为标准流程作为统一解决方案,例如医院要发布一站式在线挂号服务的情况,
 该需求大体可以通过“部门信息获取”,“专家信息获取”,
 “注册资源获取”,“注册”和“账单支付”五个功能模块(子服务)对整个业务的服务流程进行建模,可将此作为标准流程构建在跨界
 服务平台。从而实现快速接入流程,去除繁琐的冗余工作,节约开发成本,
 对外暴露统一的标准流程,方便用户,提升用户体验,用户在调用执行标准流程时系统自动映射到实体流程中。


\section{国内外研究现状}

“云大物移智”等新一代信息技术的发展与应用,使得人类的认知扩大、能力增强,
也将重新定义传统边界。“跨界”即突破原有界限,实现界内和界外资源的整合与协作。
跨界服务将跨越不同行业、组织、价值链等边界的服务进行深度融合和模式创新,为用
户提供多维度、高质量、富价值服务,这不仅是技术发展的必然,也是现代服务业发展
的重要创新途径。

相比传统的服务集成,跨界服务融合需开展模式、生态、环境、质量、价值等多维
深度融合,具有极大挑战。目前国内外依然缺少跨界服务本质规模与模式认知、设计与
管理方法、质量管理与价值工程等方面的系统研究,也缺少相关工程方法和支撑载体。
从以下四个方面综述国内外的相关研究:

(1) 服务模式创新是推动现代服务业快速发展的重要因素。近年来,国际上出现
了以Artifact 为中心的商业流程建模方法、基于商业交易过程的商业模式分析框架等,
对服务模式进行了分析和建模;国内浙江大学首次提出了跨界服务概念及其3C 特点。
但总体来看,目前学术界在跨界服务本质规律认知、模式定量分析与评估等方面仍处于
探索阶段。

(2) 跨界服务设计关注如何获取和分析角色多元的用户真实需求,并根据用户需
求进行服务架构、流程和接口等生态设计。IBM 研究院、北京大学、武汉大学等单位的
研究团队在服务需求建模、服务设计和服务互操作性管理等方面具有较好的研究积累,
做出了一系列代表性工作如SOMA-ME、RGPS 需求元建模框架、基于Tropos 的服务建模
方法等,但针对跨界服务融合的设计目前仍缺少完整、系统的支撑方法体系。

(3) 跨界服务融合需要高效、可靠的运行支撑环境,以提供服务网络的运行态支
持。目前这一领域主要有企业服务总线、企业应用集成等相关技术,北京大学、IBM、
佐治亚理工学院等在云端融合资源服务化、服务总线等方面具有较好研究积累,但这些
技术大多针对企业级运行环境,仅实现服务的结构和信息融合,难以应对跨界服务所需
的多维深度融合、动态服务网络优化、开放环境安全管控等挑战。

(4)对服务系统进行精准的能力配置以提供特定的质量与价值,并在运行时准确
感知它们的实际提供水平以做出调控和改进。IBM 研究院、荷兰阿姆斯特丹自由大学、
哈尔滨工业大学等单位的研究团队在服务质量设计与度量、服务价值建模、服务价值感
知等方面具有较好的研究积累,做出了一系列代表性工作,如VASEM、服务价值网等,
但针对跨界服务质量体系和价值工程的研究仍处于初期阶段。

\begin{table}[htb]
  \centering
  \caption{国外从事相关研究的主要机构}
  \label{tab:RelatedResearchAbroad}
    \begin{tabular}{p{4cm}|p{4cm}|p{6cm}}
      \toprule
      % & \multicolumn{1}{m{60mm}}{\heiti\centering 相关研究成果}
      \multicolumn{1}{l|}{\heiti 机构名称} & \multicolumn{1}{l|}{\heiti 相关研究内容} & \multicolumn{1}{l}{\heiti 相关研究成果}\\
      \midrule
      % IBM & 服务科学、服务工程;\\
      IBM & 服务科学、服务工程;& 提出了服务计算研究框架、服务科学、管理与工程体系\\ \hline
       Carnegie Mellon University & 服务测试、组合、推荐等服务计算关键技术 & 提出了服务集成模型、基于语义的服务发现、组合等方法\\ \hline
 University of Sydney & 服务交互模型、服务资源调度、服务质量管理 & 提出了服务质量预测方法、服务组合优化、服务信任度量等方法\\ \hline
 Cambridge Service Alliance,UK & 复杂服务系统、服务使能技术方法&提出了数据驱动复杂业务建模、复杂服务网络管理等方法\\ \hline
 American Service Research & 服务设计、服务质量管理、服务化工程&提出了智能服务设计、服务创新管理等方法\\ \hline
      % \rowcolor[gray]{.9} colortbl & 表格上色。自己看着爽而已,打印出来都是黑白的。 \\
      % threeparttable & 用来给表格添加脚注啥的很方便。 \\
      % array & 忘了用来做什么了,但似乎很重要。 \\
      \bottomrule
    \end{tabular}
\end{table}

\begin{table}[htb]
  \centering
  \caption{国内从事相关研究的主要机构}
  \label{tab:RelatedResearchInChina}
    \begin{tabular}{p{4cm}|p{4cm}|p{6cm}}
      \toprule
      % & \multicolumn{1}{m{60mm}}{\heiti\centering 相关研究成果}
      \multicolumn{1}{l|}{\heiti 机构名称} & \multicolumn{1}{l|}{\heiti 相关研究内容} & \multicolumn{1}{l}{\heiti 相关研究成果}\\
      \midrule
      % IBM & 服务科学、服务工程;\\
      浙江大学 &面向现代服务业的服务计算 &在服务计算相关CCF A 类期刊、会议以及IEEE Trans 上发表论文80余篇,累计GoogleScholar引用超过5000次;获得国家发明专利100 余项。\\ \hline
      武汉大学 &软件服务工程 &主持研制5 项软件服务 相关ISO 国际标准(已 颁布),在IEEE Transactions on Services Computing 和 ICWS 等期刊和会议发 表30 余篇相关论文。\\ \hline
      哈尔滨工业大学 &服务价值工程、 服务质量管理 &提出了价值知觉的服 务工程方法体系,在 IEEE Transactions on Services Computing 和 ICWS,、ICSOC 等相关期 刊和会议上发表30 余 篇相关论文。\\ \hline
      北京邮电大学 &网络服务与智能服务平台 &在IEEE/ACM Transactions 期刊和 CCF A 类会议上发表学 术论文50 余篇,ESI 高 被引文章4 篇,获得国 家发明专利60 余项。\\ \hline
      阿里研究院& 电子商务服务及服务中间件& 探索了电商服务模式,研究数据时代的经济范式,制定了电商服务支撑平台的若干标准,突破了服务组合、服务流程、服务导出等相关的关键技术。\\ \hline
      % \rowcolor[gray]{.9} colortbl & 表格上色。自己看着爽而已,打印出来都是黑白的。 \\
      % threeparttable & 用来给表格添加脚注啥的很方便。 \\
      % array & 忘了用来做什么了,但似乎很重要。 \\
      \bottomrule
    \end{tabular}
\end{table}



关于本文的主要论点:参考服务与参考流程,即服务的映射问题和流程的映射问题,是本文提出的创新性论点
目前国内外暂时没有找到相关研究,但下文会提到,本文把服务和流程的映射问题转化为了nlp领域相对较为成熟
的命名实体识别问题,因此本节接下来简要介绍一下关于命名实体识别的国内外研究状况。

命名实体识别(NER)的任务是识别文本范围内提及的命名实体,并将其分类为预定义的类别,
例如人名,位置,组织等。命名实体是一个单词或短语,
命名实体的示例包括一般域中的组织名称,人员名称和位置名称,生物医学领域中的基因,蛋白质,药物和疾病名称,
NER是将文本中的命名实体定位和分类为预定义实体类别的过程,如图\ref{fig:nerp2}。NER充当各种自然语言应用程序(例如问答系统,文本摘要和机器翻译)的基础。
尽管早期基于统计规则的NER系统成功地产生了不错的识别精度,但是在精心设计规则或功能时,它们通常需要大量的人工。
近年来,通过连续实值矢量表示和通过非线性处理的语义合成的支持,深度学习已被应用于NER系统中,从而产生了最优异的性能。

NER的演变过程,如图\ref{fig:nerp}所示:MUC-6首次使用“命名实体”(NE),
其任务是识别文本中的组织名称,人名和地理位置以及货币,时间和百分比表达式。
自从MUC-6以来,人们对NER的兴趣越来越高,各种测评任务为此主题投入了很多精力。关于问题的定义,
Petasis等限制了命名实体的定义:“ NE是专有名词,充当某物或某人的名称”\cite{petasis2000automatic},
关于NER中应用的技术,主流方法主要有四种:1)基于规则的方法,由于它们依赖手工制定的规则,因此不需要带标注的数据; 
2)无监督学习方法,该方法依靠无监督算法而无需手工标记训练样本; 3)基于特征的监督学习方法,该方法依赖于监督学习算法并经过精心的特征设计; 
4)基于深度学习的方法,该方法以端到端的方式自动从原始输入中发现分类和检测所需的表示形式(特征)。

\begin{figure}[htbp]
  \centering
  \includegraphics[scale=1]{./images/nerp2.jpg}
  \caption{命名实体识别流程}
  \label{fig:nerp2}
\end{figure}

\begin{figure}[htbp]
  \centering
  \includegraphics[scale=1]{./images/nerp.jpg}
  \caption{命名实体识别技术研究发展趋势}
  \label{fig:nerp}
\end{figure}

