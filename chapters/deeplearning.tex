\chapter{基于深度学习的用户语义理解}

\section{语义理解任务描述与解决流程}
\subsection{问题描述}
跨界服务平台内服务的智能调用实现过程中,语义理解是关键。系统接收的是用户输入的一句有目的性的话,系统在语义理解的过程中,
首先要识别用户的意图,根据
用户意图匹配相应的服务以及匹配该服务要执行的操作,这两者均可被视为文本分类问题,可以用深度学习的分类算法解决,
再本文被称作服务分类和接口分类任务;找到匹配的服务以后,在服务执行前
必要的执行参数,可以从用户输入的语句中提取,这将被看作语义槽填充问题,可以用序列标注算法解决,
将词语序列x=[$x_{1}$,$x_{2}$,\dots,$x_{T}$]映射到相应的插槽标签序列y=[$y_{1}$,$y_{2}$,\dots,$y_{T}$],再本文被称作参数提取任务。

以调用火车票信息查询服务为例,来解释跨界服务平台内服务分类,接口分类和参数提取。
用户在进入跨界服务平台后,输入“查询成都前往杭州的火车票”,跨界服务平台内
的语义理解模型识别出该语义对应平台内部的<train服务>,接口类型为<query查询>,语义槽(即服务参数)为<startCity起始地>成都和<endCity目的地>杭州。


\begin{figure}[htbp]
    \centering
    \includegraphics[scale=0.5]{./images/liucheng.png}
    \caption{语义理解算法流程}
    \label{fig:questiondesc}
  \end{figure}

\subsection{算法描述}
基于深度学习的语义理解算法,首先将用户输入的句子利用分词工具转化为词语序列,再用word2vec将词语向量化,
固定句子长度后,得到结构化的表示形式x=[$x_{1}$,$x_{2}$,\dots,$x_{T}$]输入网络中,流程如图\ref{fig:questiondesc}所示。
首先利用服务分类器得到系统内与用户意图相匹配的服务,再通过接口分类器和参数提取器分别做文本分类和语义槽填充处理,最后将三项任务得到
的结果提交至服务执行引擎做服务调用,算法详细描述如表\ref{tab:suanfa1}。

\begin{table}[htb]
  \centering
  \caption{基于深度学习的语义理解算法描述}
  \label{tab:suanfa1}
\begin{tabular}{p{150mm}}
\toprule
\textbf{算法1:}基于深度学习的语义理解\\
\textbf{输入:} $L_0$=\{($S_i$,$y_i^d$,$y_i^i$,$y_i^s$)\},i $\in$ [1,n]\\
\textbf{输出:} $Model_{service}$,$Model_{interface}$,$Model_{slot}$\\
\hline
\textbf{过程描述:}模型在训练时,输入的$S_i$为有用户意图的语句,经过分词和向量化
处理后得到$E_i=(e_1,e_2,\dots,e_T)$,T为词语个数。模型参数初始化、批处理的数量
iterations = M 和每一批的样本数 batchsize、迭代次数 epoch = N 和当前迭代次数 i = 0\\
\textbf{while} i < N or 模型的性能达到终止条件 \textbf{do}\\
\qquad \qquad \textbf{for} j = 1,\dots ,M \textbf{do}\\
\qquad \qquad \qquad \qquad 随机抽取 batchsize 个训练集数据,前向传输在当前网络权值和输入下网络的输出\\
\qquad \qquad \qquad \qquad 反向传输调整模型参数\\
\qquad \qquad \textbf{end for}\\
\qquad \qquad 计算损失 Loss,更新梯度和模型的参数\\
\textbf{end while}\\
用训练好的领域分类模型对测试样本进行预测,加载训练好的相应领域的意图识别模型
和语义槽填充模型,对意图和语义槽进行预测,计算准确率 Accuracy、损失 Loss\\
\bottomrule
\end{tabular}
\end{table}

\section{预处理}
\subsection{序列化}
% 结巴分词,word2vec
% 用户不需要输入完整的句子
神经网络的输入是一个向量序列,但用户输入是一个完整的句子,因此首先需要把句子序列化,之后使用word2vec将词语向量化。
本文采用结巴分词来做序列化的处理,
结巴分词分了三种模式,准确模式试图将句子切成最贴切的句段,适用于文本分析;完全模式会从句子中获取所有可能的单词,速度快但不精确;
基于准确模式的搜索引擎模式试图将长词切成几个短词,从而提高查全率,适用于搜索引擎。
结巴分词的算法依据主要是基于前缀字典结构来实现高效的词图扫描,为所有可能的单词组合构建有向无环图(DAG),
使用动态规划根据单词频率查找最可能的组合。

本文在实际处理中发现文本的预处理特别重要,如果句子中包含了太多无关词(这些词被称作停用词),算法性能会受影响,因此本文
在文本预处理时引用了停用词过滤。比较常见的中文停用词如“的”、“在”以及语气词等,他们的存在对语义理解没有贡献,相反如果分词工具使用他们错误的分词,
会降低模型准确率,因此本文利用网上已有的中文停用词词典在结巴分词工具处理前对用户语句做了停用词过滤。

\subsection{标签向量化}
服务分类和接口分类均属于多类别分类问题,训练样本的标签均采用one-hot编码,本文筛选了跨界服务平台中用户使用较多的八类服务,接口类型为这八类
服务下所有接口的合计,如query,order,play等。对于参数提取任务,像序列标注任务一样,使用BIO标注方式标记参数语义槽,根据类型不同分为
“B-X”、“I-X”和“O”,如图\ref{fig:yuyicao}所示。
\begin{figure}[htbp]
  \centering
  \includegraphics[scale=0.5]{./images/yuyicao.png}
  \caption{参数语义槽标注}
  \label{fig:yuyicao}
\end{figure}


\section{基于ATT-CLSTM的服务分类模型}
实现服务智能调用的第一步要根据用户意图匹配相应的服务,这可以被转化为文本分类问题。
LSTM
是一种特殊的RNN,能够学习长期依赖关系,理解关键字出现的顺序,擅长处理文本序列的问题,同时,
它通过增加门控机制来过滤信息,解决了长距离依赖的问题。 另一方面,在某种程度上也避免了梯度消失和梯度爆炸的问题。
CNN
多用于图像领域,由于卷积核的存在,能够提取出词与词之间的隐藏的语义信息,捕获局部相关性。
尽管CNN可以在许多任务中很好地表现,但它最大问题之一是卷积核大小的在训练时固定,不具有很好的泛化能力。一方面,无法对更长的序列信息进行建模,
另一方面,超参数卷积核大小的调整也增加了工作量,池化层的处理也会丢失了部分结构性信息,因此很难在文本中发现复杂的模式。
为了
能让卷积层从词嵌入向量中提取更高级的语义表示,我们在模型中引入attention层.这是因为在传统的CNN中,有效地编码长期上下文信息和非连续词之间的相关性并不容易,
它仅考虑在字符串上连续的n-gram信息,因此忽略了非连续词之间的某些长距离相关性,而这种相关性在许多语言中都起着重要作用。例如,用户输入“帮我挂
明天上午邵逸夫医院的号”,这里“挂”和“号”显然应该组合在一起成词才能具有完整正确的语义信息。NLP中有关注意力的大多数现有研究都集中在对不同模式之间的相关性进行建模,
因此在本文中,我们将attention层放在卷积层之前,
用于自动捕获长期上下文信息和非连续词之间的相关性,而无需任何外部语法信息。

综上所述本节将三者结合各取所长, 采用ATT-CLSTM模型\ref{fig:cnn-lstm}用以解决服务分类问题,下面将详细介绍该模型:

\begin{figure}[htbp]
    \centering
    \includegraphics[scale=0.5]{./images/cnn-lstm.jpg}
    \caption{服务分类模型}
    \label{fig:cnn-lstm}
  \end{figure}
  (1)输入层与embedding层

  用户输入的语句通过结巴分词工具得到中文词的模型即可完成输入层的任务到达embedding层。之后将文本序列转换为矢量形式的数字表示,word2vec是最常用的方法之一。
  在本实验中,我们选择通过C-BOW基于维基百科中文语料库预训练得到的word2vec模型,通过使用预训练词向量模型可以有效地减少人工神经网络的过拟合。
  由于神经网络的输入长度在本模型中是固定的,加之用户输入的语句长度通常有限,结合数据集中最长句子拆分以后的序列长度考量,我们决定把最大长度设为32.
  同时word2vec模型的输出的词向量维度为128维,因此每一个需要匹配系统内服务的用户输入的句子就构成了一个32×128的二维矩阵E=[$e_{1}$,$e_{2}$,\dots,$e_{32}$],
  其中$e_{i}$=[$e_{i1}$,$e_{i2}$,\dots,$e_{i128}$]是一个中文词经过word2vec处理的向量表示。

  (2)attention层

  如图\ref{fig:att-cnn}所示,是注意力层的主要结构,在输入层和卷积层之间引入了注意层.具体而言,注意力层是为每个词创建上下文向量,
  上下文向量与词向量串联在一起,作为新的词表示形式将被输入到到卷积层。 
  注意机制的思想是在推导$x_{i}$的上下文向量$g_{i}$时将注意力集中在特定的重要单词上,图\ref{fig:att-cnn}中的红色矩形代表$x_{i}$的上下文向量$g_{i}$,
  这种机制确定了在预测服务类别时哪些词应比句子上的其他单词获得更多的关注。
  注意机制是一个附加的多层感知器,它与ATT-CLSTM的所有其他组件共同训练。
  注意力机制生成的上下文向量$g_{i}$由以下公式得到:
  \begin{equation}
  \mathbf{g}_{i}=\sum_{j \neq i} \alpha_{i, j} \cdot \mathbf{x}_{j}
\end{equation}
其中,$\alpha_{i, j}$是注意力权重值,其计算方式如下:
\begin{equation}
\alpha_{i, j}=\frac{\exp \left(\operatorname{score}\left(\mathbf{x}_{i}, \mathbf{x}_{j}\right)\right)}{\sum_{j^{\prime}} \exp \left(\operatorname{score}\left(\mathbf{x}_{i}, \mathbf{x}_{j^{\prime}}\right)\right)}
\end{equation}
\begin{equation}
\text { score }\left(\mathbf{x}_{i}, \mathbf{x}_{j}\right)=v_{a}^{\top} \tanh \left(W_{a}\left[\mathbf{x}_{i} \oplus \mathbf{x}_{j}\right]\right)
\end{equation}
词与词之间的score值由感知器计算得到,$v_{a},W_{a}$为MLP权重参数,以此来度量词与词之间的相关性,得分越高表示相关性越强。
经过这一层的处理,用户输入的句子被转换为32×256的二维矩阵A=[$a_{1}$,$a_{2}$,\dots,$a_{32}$],
其中$a_{i}$=[$a_{i1}$,$a_{i2}$,\dots,$a_{i256}$]是一个中文词经过attention处理的向量表示。

\begin{figure}[htbp]
  \centering
  \includegraphics[scale=0.4]{./images/attcnn.jpg}
  \caption{服务分类模型中attention层}
  \label{fig:att-cnn}
\end{figure}
  (3)卷积层

  由于卷积核的存在,卷积层能够提取出词与词之间的隐藏的语义信息,捕获局部相关性。这里我们仅设置了单层的卷积层,用于捕获序列局部相关信息并对输入数据的降维。
  设置不同初始化权重的多个卷积核可以提高模型的学习能力,本模型共设置了64个卷积核,卷积核在文本矩阵上移动以提取特征。在文本处理领域CNN的卷积操作常被设置
  为一维卷积,这是因为直观上来看沿着词向量所在的维度不具有拆分性,因此卷积核总是沿着词与词之间的维度扫描提取特征,对m×n的矩阵用高为h的卷积核做一维卷积
  会得到一个高为m-h+1,宽为1的长条状矩阵。基于此,本文卷积核宽度与attention输出层
  一致为256,高度设置了[2,3]两种选择以期获得不同视角的语义信息,所以本层的输出结果大小随卷积核的尺寸不同会不一致,即会得到64个大小不一的矩阵称为feature maps.
  卷积操作的计算公式如下:
  \begin{equation}
    c=ReLU(W_{c}A+b_{c})
    \end{equation}
    这里选用ReLU作为非线性激活函数,减少参数之间的相互依赖性避免梯度消失和过拟合等问题,c为一个卷积核做一次卷积运算所得值。

   (4)pooling层

    如上层所述,卷积层的输出结果大小随卷积核的尺寸不同会不一致,因此池化层有统一维度的作用,
    同时可以降维降低计算复杂度,保留核心特征,减少模型参数防止过拟合提高模型泛化能力。
    由于卷积核卷积操作得到的结果是长条状,因此本文采用k-max pooling池化方法,k是超参数,经实验调整为4。
    经过池化层处理后,得到4×64的特征矩阵作为lstm层的输入。
    从64维的角度来看,每一维是一种高层语义特征的提取。

    (5)lstm层

  直观上,文本分类是顺序信息处理的过程。但是,从卷积神经网络以并行方式获得的特征序列不包含序列信息,LSTM专为顺序建模而设计,
  可以进一步从CNN获得的特征序列中提取上下文信息,lstm内部核心计算过程包括:$f_{t}=σ(W_{f}\cdot[h_{t-1},x_t]+b_{f})$,
  $C_{t}=f_{t} * C_{t-1}+i_{t} * \tilde{C}_{t}$,$o_{t}=\sigma\left(W_{o}\left[h_{t-1}, x_{t}\right]+b_{o}\right)$,
  $h_{t}=o_{t} * \tanh \left(C_{t}\right)$。
    其中,$x_t$为从池化层输入的变量,其他字符代表的是lstm内部计算的中间过程变量不再赘述,得到的所有隐层信息$h_t$被输入到下一层。
  (6)优化层与决策层
    由于本模型层度较深参数较多,为防止训练时发生过拟合,在lstm层后接入dropout(未在图中表示),最后利用softmax做分类决策。

实验结果表明,矩阵分别经过注意力卷积层和LSTM层后,可以获得较高程度提取的语义信息。



\section{基于Trans-CLSTM的接口分类模型}
接口分类和服务分类本质上一样都可以视为文本分类问题,我们在处理接口分类问题时没有生搬硬套服务分类模型,
而是希望改进以提升性能,因此做了不同的尝试,以期寻找到最适合跨界服务领域的文本分类模型。
CNN在许多自然语言处理任务中都很有效,由于常规的CNN具有许多几何上固定的结构 ,CNN不可避免地面临适应非规则形状特征的挑战。
例如卷积层将卷积层形状限制为n-gram,而pooling层会使用solid chunks提取高层语义特征,
这些固定的结构为语义表示带来了两个限制:所有卷积核和块都是连续的且形状固定的,
这使CNN难以处理某些复杂的情况,例如非连续或过大的特征,换句话说,它处理不了不符合卷积核形状的特征,并且传统的CNN无法主动进行调整以适应特征的转换。
在训练中他们倾向于记住数据集中各种形式的特征,但不了解这些特征的经过变换以后的形状。举个例子,
给定短语“不那么好”,传统的卷积网络很难直接捕获非连续特征“不...好”,也很难从其他变换形式中识别出,例如“不太好”,这些均为“不好”这个特征的变换形式。

本节采用了端到端的文本分类模块,修改了卷积神经网络中的传统卷积层,称作可变换卷积层,该模块可以增强CNN对变换特征的捕捉和建模能力。
可变换卷积引入位置偏差信息$\Delta p$的概念,将位置偏差信息$\Delta p$添加到卷积核的采样位置,
这样卷积核的采样位置会因为$\Delta p$而变化,达到捕获复杂特征的目的。
$\Delta p$被设计为神经网络自动学习,而无需额外的信息或人工监督。
同时由于lstm能够学习长期依赖关系,理解关键字出现的顺序,擅长处理文本序列的问题.我们把lstm拼接在池化层之后对语义信息做进一步的处理,
接口分类的模型如图\ref{fig:tran-cnn-lstm},下面将详细介绍该模型:

\begin{figure}[htbp]
  \centering
  \includegraphics[scale=0.4]{./images/tran-cnn-lstm.jpg}
  \caption{接口分类模型}
  \label{fig:tran-cnn-lstm}
\end{figure}
(1)输入层与embedding层

该层与服务分类模型做相同处理,不再赘述。需要注意的是与服务分类模型有一处不同,我们选择输出维度为256维的词向量模型,序列长度仍为32,
因此每一个需要匹配系统内服务的用户输入的句子就构成了一个32×256的二维矩阵E=[$e_{1}$,$e_{2}$,\dots,$e_{32}$],
其中$e_{i}$=[$e_{i1}$,$e_{i2}$,\dots,$e_{i256}$]是一个中文词经过word2vec处理的向量表示。

(2)可变换卷积层

可变换卷积层是本模型的核心模块,该模块具有可学习的形状以适应特征的变化。 
通常,卷积核的形状因为是超参数从一开始就是固定的,但是可变换模块将学习到的位置偏差信息添加到卷积核上,从而使其形状灵活,特征变换适应性增强。 
位置偏差信息由动态部分和静态部分组成,在预测阶段,动态偏差与当前输入有关,其值可从当前输入中经过计算主动获取以捕获特征变换信息。 
相反,静态偏差值像其他权值一样通过反向传播进行更新,并在预测阶段保持不变,这描述了模型从训练集中学到的特征信息的基本分布。
对于传统CNN的卷积核大小是固定,卷积核在对输入图像扫描过程中每一次卷积操作的范围是固定的,如图\ref{fig:tansconv}的左图所示,一次普通的卷积运算可以用以下公式表示:
\begin{equation}
  y=\mathbf{w}\mathbf{x}
  \end{equation}
  将上式展开计算卷积核中每一个元素,公式被转换为:
  \begin{equation}
    \mathbf{y}\left(p_{0}\right)=\sum_{p_{i} \in \mathbf{C}} \mathbf{w}\left(p_{i}\right) \cdot \mathbf{x}\left(p_{0}+p_{i}\right)
    \end{equation}
其中,C为本次卷积核采样的全集,$p_{0}$是卷积运算后本次卷积在结果特征图中的位置,$p_{i}$枚举了本次卷积核采样的全集中的所有位置,并且
表示与$p_{0}$的距离。

在可变换卷积(transformable convolution)中,一次卷积运算的采样全集C被分为两部分,分别用$\mathbf{D}_{c} \subset \mathbf{C}$和
$\mathbf{S}_{c} \subset \mathbf{C}$来表示,$\mathbf{D}_{c}$是与当前输入信息相关的动态位置偏移信息,$\mathbf{S}_{c}$是训练时得到的
静态位置偏移信息(可视为对位置偏移信息的基础建模)。然后这两部分在计算时分别被加在卷积核的采样位置,
分别记为$\Delta p_{i}^{\mathbf{D}_{c}}$和$\Delta p_{i}^{\mathbf{S}_{c}}$,
这样卷积的计算公式就变成了:
\begin{equation}
  \mathbf{y}\left(p_{0}\right)= \sum_{p_{i} \in \mathbf{D}_{c}} \mathbf{w}\left(p_{i}\right) \cdot \mathbf{x}\left(p_{0}+p_{i}+\Delta p_{i}^{\mathbf{D}_{c}}\right) +\sum_{p_{i} \in \mathbf{S}_{c}} \mathbf{w}\left(p_{i}\right) \cdot \mathbf{x}\left(p_{0}+p_{i}+\Delta p_{i}^{\mathbf{S}_{c}}\right)
\end{equation}
现在,$\mathbf{D}_{c}$和$\mathbf{S}_{c}$的引入,卷积核的采样位置能够做到动态的变化,实现了重新分布,而不是像以前一样是固定的矩形。
考虑到$\Delta p_{i}^{\mathbf{D}_{c}}$和$\Delta p_{i}^{\mathbf{S}_{c}}$往往是小数,因此$\mathbf{x}()$函数的计算采用线性插值的方法:
\begin{equation}
\mathbf{x}(p)=\sum_{q} K(p, q) \cdot \mathbf{x}(q)
\end{equation}
这里p代表$p_{0}+p_{i}+\Delta p_{i}^{\mathbf{D}_{c}}$或者$p_{0}+p_{i}+\Delta p_{i}^{\mathbf{S}_{c}}$,q枚举了输入矩阵的所有位置,
K是线性插值运算的核:
\begin{equation}
K(p, q)=\max (0,1-|p-q|)
\end{equation}
图\ref{fig:tansconv}显示了可变换卷积的机制,顶部的动态位置偏差信息$\Delta p_{i}^{\mathbf{D}_{c}}$
是通过一个自定义的卷积层从当前输入序列中计算得到的,
其值被添加到$\mathbf{D}_{c}$中的采样位置,
由于从输入计算生成,因此它们的值会根据当前输入的特征动态变化,以适应当前的特征变换。 
底部的静态偏差$\Delta p_{i}^{\mathbf{S}_{c}}$是常量,在训练阶段通过反向传播进行更新,并在预测时保持不变,
因此,静态偏差描述了从数据集中学到的基本特征分布规则。
本层共设置了128个卷积核,宽度与词向量长度一致,高度分为[2,3]不等,
选用ReLU作为非线性激活函数,本层输出将得到128个大小不一的长条状feature maps.

\begin{figure}[htbp]
  \centering
  \includegraphics[scale=0.4]{./images/tansconv.jpg}
  \caption{接口分类模型中的transformable conv层}
  \label{fig:tansconv}
\end{figure}
(3)输出层

由于卷积核卷积操作得到的结果是长条状,因此本文采用k-max pooling池化方法,k是超参数,经实验调整为4。
经过池化层处理后,得到4×64的特征矩阵作为lstm层的输入。
从64维的角度来看,每一维是一种高层语义特征的提取。
直观上,文本分类是顺序信息处理的过程。但是,从卷积神经网络以并行方式获得的特征序列不包含序列信息,LSTM专为顺序建模而设计,
可以进一步从CNN获得的特征序列中提取上下文信息。由于本模型层度较深参数较多,为防止训练时发生过拟合,
在lstm层后接入dropout(未在图中表示),最后利用softmax做分类决策。

实验结果表明,transformable cnn在测试集的表现要由于传统卷积层,这是由于它对变换特征的捕捉和建模能力得到了增强。
\section{引入字向量的BLSTM-ATT-CRF参数填充模型}
找到匹配的服务和相关的接口以后,在服务执行前需要从用户的输入语句提取出必要的执行参数,
因此本任务相较于分类任务更细粒度,对语义理解的要求更高,
这可以被看作命名实体识别问题,可以用序列标注算法提取执行参数。
对一个用户输入语句进行分词得到序列,如果分词操作出错,将丢失重要的语义信息,直接影响实体边界的预测,序列标注任务也不可能成功。
因此单纯依赖结巴分词工具分词以后获取输入序列信息使任务存在瓶颈,需要引入其他的embedding手段来降低分词错误带来的干扰,
我们选择在本模型中加入字符级的编码。
对于许多序列标记任务,访问过去(左)和将来(右)上下文都是有帮助的。但是,LSTM的隐藏状态$h_t$仅从过去获取信息,对右侧数据一无所知。 
双向LSTM(BLSTM)基本思想是将每个词向前和向后编码为两个单独的隐藏状态,以分别捕获过去和将来的信息,
然后将两个隐藏状态连接起来以形成最终输出,因此本模型特征提取部分主要依赖BLSTM。
同时我们选择条件随机场作为解码器,CRF考虑邻域标记之间的相关性,并针对给定输入语句联合解码得到最佳标记链。
综上所述,本节采用引入字向量的BLSTM-ATT-CRF模型\ref{fig:blstm-att-crf}解决参数填充问题,下面将详细介绍该模型:


\begin{figure}[htbp]
  \centering
  \includegraphics[width=15cm]{./images/blstm-att-crf.jpg}
  \caption{参数填充模型}
  \label{fig:blstm-att-crf}
\end{figure}

(1)输入层与embedding层

与前两个模型一致,首先,通过结巴分词工具得到词序列完成输入层的工作。但是,本模型除了词嵌入处理外,还引入了字符向量嵌入。设
$\mathbf{e}_{i}^{w}$表示用户输入语句词序列中第i个词的向量,$\mathbf{e}_{i}^{c}$指用户输入语句词序列中第i个词中所有字向量的表示,
则embedding层第i个词向BiLSTM层输入可表示为:
\begin{equation}
  \mathbf{x}_{i}=[\mathbf{e}_{i}^{w} ;\mathbf{e}_{i}^{c}]
\end{equation}
其中,我们会用另一个双向LSTM(图中未标出),输入为一句话的字向量序列,对于单词$w_i$的字向量$c_t(i,1),\dots,c_t(i,len(i))$(len(i)为词语的长度),
可以得到$\overrightarrow{\mathbf{h}}_{t(i, 1)}^{c}$,\dots,$\overrightarrow{\mathbf{h}}_{t(i, len(i))}^{c}$
和$\overleftarrow{\mathbf{h}}_{t(i, 1)}^{c}$,\dots,$\overleftarrow{\mathbf{h}}_{t(i, len(i))}^{c}$ ,上式中的$\mathbf{e}_{i}^{c}$由以下公式得出:
\begin{equation}
  \mathbf{e}_{i}^{c}=[\overrightarrow{\mathbf{h}}_{t(i, len(i))}^{c} ; \overleftarrow{\mathbf{h}}_{t(i, 1)}^{c}]
\end{equation}

(2)BiLSTM-attention层

嵌入层的向量输入到双向LSTM网络中后,对每一个词$\mathbf{e}_{i}^{w}$,前向LSTM输出带有从左往右语义的向量$\overrightarrow{\mathbf{h}}_{i}$,
后向LSTM输出带有从右往左语义的向量$\overleftarrow{\mathbf{h}}_{i}$,我们将这两个向量拼接得到$\mathbf{h}_{i}=[\overrightarrow{\mathbf{h}}_{i} ;\overleftarrow{\mathbf{h}}_{i}]$
并对$\mathbf{h}_{i}$做self attention处理。
attention层主要运算工作是计算词与词之间的语义相似度,这里代表一个词$\mathbf{e}_{i}^{w}$的向量便是$\mathbf{h}_{i}$,则词i与词j之间的attention权重$α_{i,j}$可由下式得到:
\begin{equation}
\alpha_{t, j}=\frac{\exp \left(\operatorname{score}\left(\mathbf{h}_{i}, \mathbf{h}_{j}\right)\right)}{\sum_{k} \exp \left(\operatorname{score}\left(\mathbf{h}_{i}, \mathbf{x}_{k}\right)\right)}
\end{equation}
\begin{equation}
  \operatorname{score}(\mathbf{h}_{i}, \mathbf{h}_{j}))=\tanh \left(\mathbf{W}_{a}\left[\mathbf{h}_{i} ; \mathbf{h}_{j}\right]\right)
\end{equation}
attention权重由带有参数$\mathbf{W}_{a}$的感知器计算得到,以此来表示在理解当前词i时应该放多少注意力在词j上:
\begin{equation}
  \mathbf{g}_{i}=\sum_{j=1}^{N} \alpha_{i, j} \mathbf{h}_{j}
\end{equation}
于是我们得到了拥有全局视野的词i的向量表示$\mathbf{g}_{i}$,将它与$\mathbf{h}_{i}$拼接得到向量$\mathbf{a}_{i}=[\mathbf{g}_{i} ;\mathbf{h}_{i}]$作为CRF层的输入。

(3)CRF层

在模型的最后,添加了CRF层以希望解码得到所有可能的标记路径中的最佳标记。 
t步长得到的标签预测向量可由$y_{t}^{s}=\operatorname{softmax}\left(W^{s} \overleftarrow{a}_{t}+b^{s}\right)$
得到,
引入标记转换矩阵$\mathbf{T}$,其中$T_{i,j}$表示上下连续两个单词中上一个被标记i以及下一个被标记j的转换分数,
而$T_{0,j}$表示标记j作为序列初始词标记的分数,该转换矩阵将做为模型的参数被训练。 
句子$\mathbf{X}$的被标记为$\mathbf{y}=(y_1,y_2,\dots,y_n)$的得分可由下式得出:
\begin{equation}
  s(\mathbf{X}, \mathbf{y})=\sum_{i=1}^{n}\left(T_{y_{i-1}, y_{i}}+P_{i, y_{i}}\right)
\end{equation}
$P_{i, y_{i}}$表示i位置softmax输出中标签$y_{i}$的概率,之后可以通过softmax函数推导出$\mathbf{X}$的被标记为$\mathbf{X}=(y_1,y_2,\dots,y_n)$的概率:
\begin{equation}
  p(\mathbf{y} \mid \mathbf{X})=\frac{e^{s(\mathbf{X}, \mathbf{y})}}{\sum_{\tilde{\mathbf{y}}} e^{s(\mathbf{X}, \tilde{\mathbf{y}})}}
\end{equation}
在训练时,模型的目标是使正确标注信息的概率最大。
设$y^{s}$表示槽的真实标签,$\hat y^{s}$表示槽的预测标签,插槽slot的损失函数采用max-margin(hinge loss):
\begin{equation}
\Delta\left(y^{s}, \hat{y}^{s}\right)=\sum_{t=1}^{T} 1\left\{y_{t}^{s} \neq \hat{y}_{t}^{s}\right\}
\end{equation}
对整个标签序列来说,损失函数定义如下:
\begin{equation}
L=\max (0, s(X,\hat{l}^{\hat{s}})+\Delta(y^{s}, \hat{y}^{\hat{s}})-s(X,y^{s}))
\end{equation}






\section{本章小结}
本章介绍了对于语义理解问题传统深度学习的解决方案,首先对本文面临的语义理解任务以及算法流程做了详细描述,
之后针对服务分类、接口分类和参数提取三个任务,分别给出了基于词向量的解决方法:基于ATT-CLSTM的服务分类模型,
基于Trans-CLSTM的接口分类模型和引入字向量的BLSTM-ATT-CRF参数填充模型。